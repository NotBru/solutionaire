% VLDB template version of 2020-03-05 enhances the ACM template, version 1.7.0:
% https://www.acm.org/publications/proceedings-template
% The ACM Latex guide provides further information about the ACM template

\documentclass{article}
\usepackage{listings}
\usepackage[margin=20mm]{geometry}
\usepackage{graphicx}
\usepackage{amsfonts}
\usepackage{amsmath}
\usepackage{physics}
\usepackage{parskip}
\usepackage{enumitem}
\usepackage{cancel}

%% The following content must be adapted for the final version
% paper-specific
\newcommand\vldbdoi{XX.XX/XXX.XX}
\newcommand\vldbpages{XXX-XXX}
% issue-specific
\newcommand\vldbvolume{14}
\newcommand\vldbissue{1}
\newcommand\vldbyear{2020}
% should be fine as it is
\newcommand\vldbauthors{\authors}
\newcommand\vldbtitle{\shorttitle} 
% leave empty if no availability url should be set
\newcommand\vldbavailabilityurl{http://vldb.org/pvldb/format_vol14.html}
\newcommand\oname{\operatorname}

\newtheorem{theorem}{Teor.}
\newtheorem{definition}{Def.}
\newtheorem{example}{Ej.}
\newtheorem{excercise}{Ejer.}

\begin{document}

\textbf{Ex. 2.1: }Is it possible to find a general formula for $p(C|A+B)$, analogous to $(2-48)$, from the product and sum rules? If so, derive it; if not, explain why this cannot be done.

\textbf{Answer:}

Sure. We've derived Bayes theorem without naming it, which is simply product rule and commutativity, so we can say
\begin{align*}
	p(C|A+B)&=\frac{p(C)\,p(A+B|C)}{p(A+B)}\\
	&=\frac{p(C)\,\left(p(A)+p(B)-p(AB|C)\right)}{p(A)+p(B)-p(AB)}.
\end{align*}

\textbf{Ex. 2.2: }Now suppose we have a set of propositions $\{A_1,\,\cdots,\,A_n\}$ which on information $X$ are mutually exclusive: $p(A_iA_j|X)=p(A_i|X)\delta_{ij}$. Show that $p(C|(A_1+A_2+\cdots+A_n)X)$ is a weighted average of the separate plausibilities $p(C|A_iX)$:
\begin{align*}
	p(C|(A_1+\cdots+A_n)X)=p(C|A_1X+A_2X+\cdots+A_nX)=\frac{\sum_ip(A_i|X)p(C|A_iX)}{\sum_ip(A_i|X)}.
\end{align*}

\textbf{Answer:}
For $i\neq j$,
\begin{align*}
	p(C|A+B)&=\frac{p(A+B|C)p(C)}{p(A+B)}\\
	&=\frac{(p(A)+p(B)-p(AB|C))p(C)}{p(A)+p(B)-p(AB|C)}.
\end{align*}
Iterating this over $\sum_iA_i$ we get
\begin{align*}
	p\left(\sum_iA_i\middle|X\right)&=\sum_ip(A_i|X).
\end{align*}

Thus, and in a similar manner to the above excercise,
\begin{align*}
	p\left(C\middle|\sum_iA_iX\right)&=\frac{p\left(\sum_iA_i\middle|CX\right)p(C|X)}{p(\sum_iA_i|X)}\\
	&=\frac{\sum_ip(A_i|CX)p(C|X)}{\sum_ip(A_i|X)}\\
	&=\frac{\sum_ip(A_i|X)p(C|A_iX)}{\sum_ip(A_i|X)}.
\end{align*}

\textbf{Ex.: }Let $A_i$ mutually exclusive ($P(A_iA_j)=P(A_i)\delta_{ij}$). Then, $P(\sum_iA_i)=\sum_iP(A_i)$.

\textbf{Answer: }

If $i\in\{1,\,2\}$, then
\begin{align}
	P(\sum_iA_i)&=P(A_1+A_2)\\
	&=P(A_1)+P(A_2)-\cancelto0{P(A_1A_2)}\\
	&=\sum_iP(A_i).
\end{align}

Then, by induction,
\begin{align}
	P\left(\sum_i^{N+1}A_i\right)&=P\left(\sum_i^NA_i\right)+P(A_{N+1})-P\left(\sum_i^NA_iA_{N+1}\right)\\
	&=\sum_i^NP(A_i)+P(A_{N+1})-\sum_i^N\cancelto0{(A_iA_{N+1})}\\
	&=\sum_iP(A_i).
\end{align}

\textbf{Ex. 2.3: Limits on Probability Values: } As soon as we have the numerical values $a=P(A|C)$ and $b=P(B|C)$, the product and sum rules place some limits on the possible numerical values for their conjunction and disjuncion. Supposing that $a\leq b$, show that the probability of the conjunction cannot exceed that of the least possible proposition: $0\leq P(AB|C)\leq a$, and the probability of the disjunction cannot be less than that of the most probable proposition: $b\leq P(A+B|C)\leq 1$. Then show that, if $a+b>1$, there is a stronger inequality for the conjunction; and if $a+b<1$ there is a stronger one for the disjunction. These necessary general inequalities are helpful in detecting errors in calculations.

\textbf{Answer: }

For the conjunction,
\begin{align}
	P(AB|C)&=P(A|C)P(B|AC)\\
	&=aP(B|AC)
	&\leq a.
\end{align}

Since $P(AB|C)=b+a-P(A+B|C)$, if $b+a>1$ then $P(AB|C)>1-P(A+B|C)=P(\overline{A+B}|C)$.

For the disjunction,
\begin{align}
	P(A+B|C)&=P(A|C)+P(B|C)-P(AB|C)\\
	&=b+a-P(AB|C)\\
	&\geq b.
\end{align}

And also, $b+a<1$ implies $P(A+B|C)<1-P(AB|C)=P(\overline{AB}|C)$.

\end{document}
\endinput
